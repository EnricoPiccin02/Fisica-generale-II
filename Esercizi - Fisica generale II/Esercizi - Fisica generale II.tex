\documentclass[a4paper]{extarticle}
\usepackage[utf8]{inputenc}
\usepackage[italian]{babel}
\selectlanguage{italian}
\usepackage[table]{xcolor}
\usepackage{xcolor}
\usepackage{circuitikz}
\usetikzlibrary{positioning, circuits.logic.US}
\usetikzlibrary {shapes.gates.logic.US, shapes.gates.logic.IEC, calc}
\tikzset {branch/.style={fill, shape = circle, minimum size = 3pt, inner sep = 0pt}}
\usetikzlibrary{matrix,calc}
\usepackage{multirow}
\usepackage{float}
\usepackage{geometry}
\usepackage{pgfplots}
\usepackage{tabularx}
\usepackage{pgf-pie}
\usepackage{tikz}
\usepackage{amsmath}
\usepackage{amssymb}
\usepackage{color, soul}
\usepackage{fancyhdr}
\usepackage{graphicx}
\usepackage{subfig}
\graphicspath{ {./img/} }
\newtheorem{theorem}{Teorema}[section]
\newtheorem{corollary}{Corollario}[theorem]
\newtheorem{lemma}[theorem]{Lemma}

% Specifiche
\geometry{
 a4paper,
 top=20mm,
 left=30mm,
 right=30mm,
 bottom=30mm
}

\pagestyle{fancy}
\fancyhf{}
\fancyhead[LO]{\nouppercase{\leftmark}}
\fancyfoot[CE, CO]{\thepage}
\addtolength{\headheight}{1em}
\addtolength{\footskip}{-0.5em}

\newcommand{\quotes}[1]{``#1''}
\renewcommand\tabularxcolumn[1]{>{\vspace{\fill}}m{#1}<{\vspace{\fill}}}
\renewcommand\arraystretch{}
\newcolumntype{P}{>{\centering\arraybackslash}X}

\title{\textbf{Università di Trieste\\ \vspace{1em}
Laurea in ingegneria elettronica e informatica}}
\author{Enrico Piccin - Corso di Fisica generale I - Prof. Vittorio Di Trapani}
\date{Anno Accademico 2021/2022 - 4 Marzo 2022}

\begin{document}

\vspace{-10mm}
\maketitle

\tableofcontents
\newpage

\noindent
\begin{center}
  4 Marzo 2022
\end{center}

\section{Analisi dimensionale}
Se si vuole misurare una distanza o una lunghezza

\newpage
\noindent
\begin{center}
  11 Marzo 2022
\end{center}
La relazione che lega spazio e velocità è
\[v = \frac{dx}{dt}\]
mentre la relazione che riguarda l'accelerazione è
\[a = \frac{dv}{dt} = \frac{d}{dt} \frac{dx}{dt} = \frac{d^2x}{dt}\]
Per esempio, se la velocità $v$ è costante, si ha che
\[\int dx = \int v dt \longrightarrow x - x_0 = v_0 \cdot t \longrightarrow \boxed{v_0 \cdot t + x_0}\]
Nel caso di un moto rettilineo uniformemente accelerato si ha che
\[\boxed{x(t) = \frac{1}{2} a t^2 + v_0 t + x_0}\]
e
\[\boxed{v(t) = at + v_0}\]

\vspace{1em}
\subsection{Esrcizio 1 - Distanza di sicurezza}
Date due automobili $\boldsymbol{a}$ e $\boldsymbol{b}$ viaggaino in un tratto rettilineo in autostrada, con la stessa velocità $v_0 = 130$ km/h. A causa di un ostacolo imprevisto, ad un certo istante l'automobile di testa $\boldsymbol{A}$ frena. Durante la frenata l'automobile $\boldsymbol{A}$ prosegue con accelerazione $a_A$ costante ... continua ...

\vspace{1em}
\noindent
\textbf{Svolgimento}: All'istante $t=0$ l'autovettura $A$ inizia a decelerare, muovendosi di moto uniformemente decelerato e dopo un tempo $\tau = 0.55$ s anche la vettura $\boldsymbol{B}$ inizia a decelerare, mentre la vettura $\boldsymbol{A}$ si arresta alla distanza di $169$ m.\\
Per conoscere l'accelerazione $a_A$ è sufficiente considerare il seguente sistema
\[
  \left\{
    \begin{array}{l}
      x(t) = -\frac{1}{2} a_A t^2 + v_0 t + x_0\\
      v(t) = -a_A t + v_0
    \end{array}
  \right.
\]
Ma saoendo che al tempo di arresto $t_A$ la velocità $v(t_A) = 0$ e $x(t_A) = l + d$ da cui si evince
\[
  \left\{
    \begin{array}{l}
      d + l = -\frac{1}{2} a_A t^2 + v_0 t + d\\
      0 = -a_A t + v_0
    \end{array}
  \right.
\]
Da cui si ottiene
\[
  \left\{
    \begin{array}{l}
      l = -\frac{1}{2} a_A t^2 + v_0 t\\
      t = \frac{v_0}{a_A}
    \end{array}
  \right.
\]
per cui si ha
\[
  \left\{
    \begin{array}{l}
      a_A = \frac{1}{2} \frac{v_0^2}{l}\\
      t = \frac{v_0}{a_A}
    \end{array}
  \right.
\]
Da cui è immediato evncere che
\[a_A = 3.85 \frac{\text{m^2}}{\text{s}^2}\]
con direzione parallela a quella della velocità e verso opposto.\\
Inoltre, sapendo che $a_A = a_B$, si determini la distanza minima di sicurezza affinché i due veicoli non si urtino. È noto che fino lo spazio percorso da $B$ di moto rettilineo uniforme è pari a
\[
  \left\{
    \begin{array}{ll}
      x_B(t) = v_0 \cdot t & \text{con } t \leq \tau\\
      x_B(t) = -\frac{1}{2} a (t - \tau)^2 + v_0 \cdot (t - \tau) + v_0 \tau
    \end{array}
  \right.
\]
Ovviamente, affinché i due veicoli non si urtino deve essere che
\[v_0 \cdot \tau < d \longrightarrow v_0 \cdot \tau = 18 m\]
Nel caso in cui l'accelerazione di $B$ sia dimezzata rispetto a quella di $A$ e la distanza di sicurezza $d$ valga $18.5$ m, si calcoli la velocità d'urto.\\
Per determinare la collisione si consideri il seguente sistema
\[
  \left\{
    \begin{array}{ll}
      x_B(t) = -\frac{1}{4} a (t - \tau)^2 + v_0 \cdot t\\
      x_A(t) = -\frac{1}{2} a t^2 + v_0 \cdot t + d\\
    \end{array}
  \right.
\]
Mettendo a sistema e, quindi, eguagliando le due equazioni si ottiene che il tempo di collisione
\[t_{u} = \frac{-a \tau + \sqrt{2a^2\tau^2 + 4ad}}{a} = 4.28 s\]
Una volta che è noto il tempo dell'urto, è possibile determinare la velocità d'urto delle due vetture, come segue
\begin{flalign*}
  v_A & = -a t_u + v_0\\
  v_B & = - \frac{1}{2} a t_u + v_0
\end{flalign*}

\vspace{1em}
\subsection{Esercizio 2 - Salto di una rampa}
Uno skateboard ...

\vspace{1em}
\noindent
\textbf{Svolgimento}: Per determinare la velocità $v_A$ dello skateboard all'inizio della rampa è necessario ruotare di $45^\circ$ il sistema di riferimento sul piano inclinato.\\
Da tale cambiamento di sistema di riferimento si ottiene
\[
  \left\{
    \begin{array}{ll}
      x_y = 0\\
      v_y = 0
    \end{array}
  \right.
\]
e
\[
  \left\{
    \begin{array}{ll}
      x'(t) = -\frac{1}{2} g \cdot \cos(45) \cdot t^2 + v_0 t\\
      v_B = - g \cdot \cos(45) \cdot t + v_0
    \end{array}
  \right.
\]
Applicando i dati noti dal problema si ottiene
\[
  \left\{
    \begin{array}{ll}
      h \cdot \sqrt{2} = -\frac{1}{2} g \cdot \cos(45) \cdot t^2 + v_0 t\\
      0 = - g \cdot \cos(45) \cdot t + v_0 \longrightarrow t = \frac{v_0}{g \cos(45)}
    \end{array}
  \right.
\]
Per cui si ottiene che $v_0$, ovverosia la velocità iniziale è proprio
\[v_0^2 = 2 h g \longrightarrow v_0 = \sqrt{2 h g} = 2.93 \frac{\text{m}}{\text{s}}\]
anche se si deve considerare il doppio di tale velocità, ottenendo
\[v_0' = 2 \cdot v_0 \cong 5,94 \frac{\text{m}}{\text{s}}\]
Adesso è possibile considerare il nuovo sistema conoscendo la vera velocità iniziale, al fine di determinare la velocità con cui raggiunge la sommità della rampa, grazie alla formula seguente
\[v^2 - v_0^2 = 2a \cdot (x - x_0) \longrightarrow v = \sqrt{v_0^2 - 2 g \cdot \cos(45) \cdot x} = 5.14 \frac{\text{m}}{\text{s}}\]
Sapendo che tale velocità è orientata a $45^\circ$ si può scomporre il moto parabolico nelle sue componenti
\[
  y : \left\{
    \begin{array}{ll}
      y(t) = - \frac{1}{2} g t^2 + v_y t + h\\
      v(t) = - g t + v_y
    \end{array}
  \right.
\]
e
\[x(t) = x_x \cdot t\]
Volendo ricavare il tempo impiegato per toccare terra si impiega il primo sistema e si estrapola
\[t = \frac{v_y + \sqrt{v_y^2 + 2gh}}{g} = 0.58 \text{ s}\]
Avendo determinato il tempo di caduta, si può calcolare lo spazio percorso sull'asse $x$ si ottiene
\[x(t) = v_x \cdot t = 3,1 \text{ m}\]








\end{document}
