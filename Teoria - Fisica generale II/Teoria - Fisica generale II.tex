\documentclass[a4paper]{extarticle}
\usepackage[utf8]{inputenc}
\usepackage[italian]{babel}
\selectlanguage{italian}
\usepackage[table]{xcolor}
\usepackage{xcolor}
\usepackage{circuitikz}
\usepackage{bm}
\usetikzlibrary{patterns,snakes}
\usetikzlibrary{decorations.markings,intersections,calc}
\usepackage{ifthen}
\usetikzlibrary{calc,patterns,angles,quotes}
\usetikzlibrary{positioning, circuits.logic.US}
\usetikzlibrary {shapes.gates.logic.US, shapes.gates.logic.IEC, calc}
\tikzset {branch/.style={fill, shape = circle, minimum size = 3pt, inner sep = 0pt}}
\usetikzlibrary{matrix,calc}
\usetikzlibrary{arrows.meta}
\usetikzlibrary{decorations.markings}
\usetikzlibrary{shapes.geometric}
\usepackage{multirow}
\usepackage{float}
\usepackage{geometry}
\usepackage{pgfplots}
\usepackage{tabularx}
\usepackage{pgf-pie}
\usepackage{tikz}
\usepackage{tikz-3dplot}
\usepackage{amsmath}
\usepackage{amssymb}
\usepackage{color, soul}
\usepackage{fancyhdr}
\usepackage{graphicx}
\usepackage{subfig}
\usepackage{physics}
%\usepackage{luamplib}
%\usepackage{mathdesign}
\usepackage[outline]{contour} % glow around text
\contourlength{1.0pt}
\graphicspath{ {./img/} }
\newtheorem{theorem}{Teorema}[section]
\newtheorem{corollary}{Corollario}[theorem]
\newtheorem{lemma}[theorem]{Lemma}

% Specifiche
\geometry{
 a4paper,
 top=20mm,
 left=30mm,
 right=30mm,
 bottom=30mm
}

\pagestyle{fancy}
\fancyhf{}
\fancyhead[LO]{\nouppercase{\leftmark}}
\fancyfoot[CE, CO]{\thepage}
\addtolength{\headheight}{1em}
\addtolength{\footskip}{-0.5em}

\newcommand{\quotes}[1]{``#1''}
\renewcommand\tabularxcolumn[1]{>{\vspace{\fill}}m{#1}<{\vspace{\fill}}}
\renewcommand\arraystretch{}
\newcolumntype{P}{>{\centering\arraybackslash}X}
\newcommand{\gear}[5]{%
    \foreach \i in {1,...,#1}
    {   [rotate=(\i-1)360/#1] (0:#2) arc (0:#4:#2) {[rounded corners=0.5pt] -- (#4+#5:#3)  arc (#4+#5:360/#1-#5:#3)} --  (360/#1:#2)
    }
}
\newcommand\dif{\mathop{}\!\mathrm{d}}

\title{\textbf{Università di Trieste\\ \vspace{1em}
Laurea in ingegneria elettronica e informatica}}
\author{Enrico Piccin - Corso di Fisica generale II - Prof. Pierluigi Monaco e Prof. Gabriele Cescutti}
\date{Anno Accademico 2022/2023 - 3 Ottobre 2022}

\begin{document}

\vspace{-10mm}
\noindent
\maketitle

\tableofcontents
\newpage

\noindent
\begin{center}
  3 Ottobre 2022
\end{center}

\section{Introduzione all'elettrostatica}
Quando consideriamo una barretta di vetro appesa ad un filo che viene strofinata su un pezzo di lana e la si avvicina ad un'altra barretta, in posizione fissa, la prima si allontana.\\
Se al posto di barrette di vetro si considerano barrette di plastica, si osserva il medesimo fenomeno di allontanamento.\\
Se, però, si considera una barretta di vetro e una di plastica, allora si ottiene un fenomeno opposto: le bacchette si avvicinano.\\
Ciò che, infatti, risulta fondamentale da capire in elettrostatica, è che la forza elettrostatica è sia \textbf{attrattiva} che \textbf{repulsiva}, a seconda della tipologia di cariche elettroniche che interagiscono.\\
Da notare, inoltre, che quando si parla di bacchette di carica positiva o negativa, si sta parlando di bacchette alle quali si sottraggono \textbf{cariche negative} oppure se ne aggiungono, in quanto gli elettroni sono le uniche particelle che si muovono.
La carica dell'elettrone è la seguente
\[\boxed{e = 1.602176634 \times 10^{-19} \text{ C}}\]
Si sta parlando, comunque, di cariche che orbitano attorno al nucleo (il quale presenta un diametro di $5.0 \times 10^{-15} \text{ m}$), mentre il diametro dell'atomo è di circa $2.0 \times 10^{-10} \text{ m}$.

\vspace{1em}
\noindent
\textbf{Osservazione}: Il fatto che gli elettroni non cadano dentro il nucleo, che presenta particelle positive come i protoni, è dettato dal principio di indeterminazione di Heisenberg, il quale afferma che, dal punto di vista quantistico, è impossibile conoscere simultaneamente con precisione sia il momento sia la posizione di una particella.

\vspace{1em}
\subsection{Conduttore}
Si consideri una sfera conduttrice in cui vi sono elettroni che tendono a distribuirsi su un lato della superficie della sfera, lasciando dall'altro lato una carica positiva.
Se a tale sfera viene collegato un cavo conduttore, gli elettroni tenderanno a percorrere tale cavo, diminuendo, di fatto, la quantità di carica negativa presente nella sfera conduttrice.

\vspace{1em}
\noindent
\textbf{Osservazione}: Come si è detto, la carica è quantizzata, e la \textbf{carica base} di un elettrone è
\[\boxed{e = 1.6 \times 10^{-19} \text{ C}}\]
che, ovviamente, presenta un segno negativo. Quando un oggetto è carico, significa che vi è uno squilibrio tra il numero degli elettroni e il numero di protoni.
Si potrebbe scrivere che la quantità di carica è data da
\[q= \left(N_p - N_e\right) \cdot e\]
dove $N_e$, $N_p$ rappresentano il numero di elettroni e di protoni.\\
Non solo, ma dato un oggetto, è possibile definire
\begin{itemize}
  \item \textbf{densità lineare di carica}, definita come $\lambda = \frac{q}{L}$;
  \item \textbf{densità superficiale di carica}, definita come $\sigma = \frac{q}{S}$;
  \item \textbf{densità volumetrica di carica}, definita come $\rho = \frac{q}{V}$.
\end{itemize}
In generale, poi, la quantità di carica totale, in elettrostatica, si \textbf{conserva}.

\vspace{1em}
\noindent
\subsection{Legge di Coulomb}
Per misurare la forza di gravità è stata impiegata la \textit{bilancia di torsione}, impiegando la forza elastica per misurare un'ulteriore forza.
Analogamente, ponendo due cariche opposte vicine le une alle altre, si misura l'angolo che viene descritto dalle due cariche nello spazio angolare: in base a tale dato, unito al fatto che è nota la forza di torsione in funzione dell'angolo stesso, si riesce a determinare la \textbf{forza di Coulomb}.
Di qui si ha che
\[\boxed{\vec{F}_{\text{a,b}} = K \cdot \frac{q_\text{A} \cdot q_\text{B}}{r_{\text{a, b}}^2} \cdot \hat{v}_{\text{a,b}}}\]
Naturalmente la forza ha una sua direzione e un suo verso, oltre che modulo, descritto dal versore.
Per la $3^{\circ}$ legge di Newton, si ha naturalmente che
\[\vec{F}_{\text{a,b}} = \vec{F}_{\text{b,a}}\]
ovvero le forze sono opposte a seconda del versore impiegato. Si ha che la costante $\epsilon_0$ è la costante dielettrica del vuoto, definita come
\[\epsilon_0 = 8.854 \times 10^{-12} \text{ } \frac{\text{C}^2}{\text{N} \cdot \text{m}^2}\]
Mentre la costante $K$ di Coulomb è 
\[\boxed{K = \frac{1}{4 \pi \epsilon_0} = 9 \times 10^{9} \text{ } \frac{\text{N} \cdot \text{m}^2}{\text{C}^2}}\]

\vspace{1em}
\noindent
\textbf{Osservazione}: La carica di $1$ C è molto elevata, in quanto se si pongono due cariche da $1$ C a distanza di $1$ m si ottiene una forza di Coulomb pari a $F_{\text{a, b}} = 9 \times 10^9$ N, che è elevatissima.\\
Non solo, si osservi che le forze attrattive agenti su un corpo si sommano, al fine di ottenere la forza risultante.

\vspace{1em}
\noindent
\textbf{Esempio 1}: Per capire se, date due cariche, risulta più significativa la forza di gravità o la forza di Coulomb, è sufficiente considerare due protoni entrambi di carica elementare $e$, posti a distanza $x$.
In particolare is ha che
\begin{itemize}
  \item La forza di Coulomb è data $F_{\text{C}, \text{a, b}} = K \cdot \frac{e^2}{x^2}$
  \item La forza di gravità è data da $F_{\text{G}, \text{a, b}} = G \cdot \frac{m_\text{p}^2}{x^2}$
\end{itemize}
Da cui si evince che il loro rapporto è dato da
\[\frac{F_{\text{C}, \text{a, b}}}{F_{\text{G}, \text{a, b}}} = \frac{e^2}{m_\text{p}^2} \cdot \frac{\text{K}}{\text{G}} = \frac{1.6 \times 10^{-38}}{1.67 \times 10^{-54}} \cdot \frac{9 \times 10^9}{6.67 \times 10^{-11}} = \frac{9}{7} \times 10^{36}\]
il che significa che la forza di Coulomb è circa $10^{36}$ volte quella di gravità.

\vspace{1em}
\noindent
\textbf{Esempio 2}: Due sfere identiche di polistirolo sono appese tramite un filo lungo $l = 30 \times 10^{-2}$ m. Le cariche delle due sfere sono incognite, ma le lor masse, invece, sono $m_1 = m_2 = 0,030 $ kg.
L'angolo descritto dal filo rispetto alla verticale è $\theta = 7^{\circ}$, è facile capire come la distanza delle due cariche sia $30 \times 10^{-2} \text{ m} \cdot \sin(7) \cdot 2 = 39 \times 10^{-2} \text{ m}$.
Non solo, ma è anche noto come
\[m g = T \cos(\theta) \hspace{1em} \text{ e } \hspace{1em} F_\text{C} = T \sin(\theta)\]
per cui si ha che
\[F_\text{C} = mg \tan(\theta)\]
Da ciò si evince come le due cariche siano
\[q_1 = q_2 = \sqrt{\frac{mg \tan(\theta) \cdot (2 l \sin(\theta))^2}{\text{K}}} = 1.46 \times 10^{-7} \text{ C}\]
che si può anche scrivere come $146$ nC.

\newpage
\noindent
\begin{center}
  4 Ottobre 2022
\end{center}
La forza di Coulomb è alla base della forza elettrostatica. Non solo, ma è fondamentale capire che, in elettrostatica, le forze possono essere sia attrattive che repulsive.\\
La carica è quantizzata e le forze elettrostatiche che agiscono su una particella possono essere sommate (secondo le regole del parallelogramma) al fine di determinarne la risultante.

\vspace{1em}
\noindent
\subsection{Campo elettrico}
Esattamente come nel caso del campo gravitazionale, anche il campo elettrico è un campo vettoriale che presenta implicite delle proprietà che attribuisce alle entità che vi interagiscono.\\
Di seguito si espone la definizione di \textbf{campo elettrico}:

% Tabella per le definizione di concetti, etc...
\vspace{1em}
\rowcolors{1}{black!5}{black!5}
\setlength{\tabcolsep}{14pt}
\renewcommand{\arraystretch}{2}
\noindent
\begin{tabularx}{\textwidth}{@{}|P|@{}}
    \hline
    {\textbf{CAMPO ELETTRICO}}\\
    \parbox{\linewidth}{Il campo elettrico, dal punto di vista vettoriale, viene definito come
    \[\vec{E} = \frac{\vec{F}}{q_0}\]
    in cui $q_0$ deve essere piccolo, ed è una carica di prova necessaria per misurare il campo elettrico, in quanto è noto che le cariche fra di loro interagiscono e si influenzano reciprocamente.\\
    Il campo elettrico è additivo, per cui al fine di conoscere il vettore campo elettrico risultante, è sufficiente sommare i vettori campo elettrico secondo la regola del parallelogramma.\vspace{3mm}}\\
    \hline
\end{tabularx}

\vspace{2em}
\noindent
\textbf{Esempio}: Si consideri una carica elettrica piccola $q_0=81 \text{ nC}$ e una forza che agisce su tale carica $\vec{F}$ ..., allora ... continua ...

\vspace{1em}
\noindent
\subsubsection{Campo elettrico generato da una carica puntiforme}
È noto che la forza di Coulomb è data dalla seguente equazione
\[\boxed{\vec F = \frac{1}{4 \pi \epsilon_0} \cdot \frac{q \cdot q_0}{r^2} \cdot \hat{v}}\]
Per cui il campo elettrico generato da una carica puntiforme è dato da
\[\boxed{\vec E = \frac{1}{4 \pi \epsilon_0} \cdot \frac{q}{r^2} \cdot \hat{v}}\]

\vspace{1em}
\noindent
Se ora si dovesse considerare il campo elettrico generato da un insieme di cariche è dato da:
\[\boxed{\vec E = \frac{1}{4 \pi \epsilon_0} \cdot \sum_{i=1}^n \frac{q_i}{r_i^2} \cdot \hat{v_i}}\]

\vspace{1em}
\noindent
\textbf{Esempio}: Data una carica $q=81 \times 10^{-9}$ C ed essendo nota la carica di un elettrone $e = 1.6 \times 10^{-19}$ C, è facile capire che il numero di elettroni persi è dato da:
\[\frac{81 \times 10^{-9}\text{ C}}{1.6 \times 10^{-19}\text{ C}} = 50.6 \times 10^{9}\]

\vspace{1em}
\noindent
\textbf{Osservazione}: In un classico esperimento R.A. Millikan (1868-1953) misure la carica dell'elettrone. L'apparecchiatura the use rappresentata schematicamente nella Figura 1.13. Un nebulizzatore produceva goccioline alcune delle quali cadevano attraverso un foro in una regione in cui era presente un campo elettrico uniform generato da due piatti paralleli carichi. Millikan era in grado di osservare una particolare gocciolina con it microscopio e di determinarne la massa misurandone la velocità limite. Egli caricava poi la gocciolina, irraggiandola con raggi X e regolava it campo elettrico in modo che la goccia rimanesse in equilibria statico sotto 1' azione delle uguali e opposte forze gravitazionale ed elettrica.

\vspace{1em}
\noindent
\subsection{Dipolo elettrico}
Avvicinando un bastoncino carico ad uno non carico, si osserva un avvicinamento dei due oggetti. Il dipolo ha un direzione ben chiara:


% \begin{figure}[H]
%   % DIPOLE - axis beneath
%   \begin{tikzpicture}
%     \def\R{0.48}
%     \def\a{2.0}
%     \def\h{0.7}
%     \coordinate (Q-) at (-\a,\h);
%     \coordinate (Q+) at (+\a,\h);
%     \coordinate (P)  at (+2.5\a,\h);
    
%     \draw[->,thick] (-1.5\a,0) -- (+3.0\a,0);
%     \draw[thick] ( 0,0.15) --++ (0,-0.3) node[below] {0};
%     \draw[thick] (-\a,0.1) --++ (0,-0.2) node[below] {$-a$};
%     \draw[thick] (+\a,0.1) --++ (0,-0.2) node[below] {$+a$};
%     \draw[thick] (2.5\a,0.1) --++ (0,-0.2) node[below] {$x$};
    
%     \draw[vector,line width=2]  (Q-) ++ (\R,0) --++ ({2(\a-\R)},0) node[midway,above] {$\vb{L}$};
%     \draw[charge-] (Q-) circle (\R) node[scale=1.0] {$-q$};
%     \draw[charge+] (Q+) circle (\R) node[scale=1.0] {$+q$};
%     \draw[vector,line width=2,Ecol] (P) --++ (0.9\a,0) node[above=2,above left=0] {$\vb{E}$};
%     \fill (P) circle (0.1) node[above=2] {P}; % node[below=2] {$x$};
%   \end{tikzpicture}
% \end{figure}

\vspace{1em}
\noindent
Allora il momento di dipolo di un dipolo elettrico si calcola come:
\[\vec p = (2aq) \cdot \hat{j}\]
mentre il campo di dipolo si determina come
\[\vec{E}(\vec{r}) = \frac{1}{4 \pi \epsilon_0} \cdot \frac{p}{r^3} \cdot \left[3 \cdot (\hat{r} \cdot \hat{p}) \hat{r} - \hat{p}\right]\]

\vspace{1em}
\noindent
\subsection{Campo elettrico sull'asse di un dipolo}
Un dipolo elettrico è composto da due carica $+q$ e $-q$ separate da una distanza $2a$. Se le due cariche sono posizionate rispettivamente in $(0,0,a)$ e $(0,0,-a)$ sull'asse $z$ (ossia l'asse del dipolo).\\

\vspace{1em}
\noindent
\subsection{Campo elettrico nel piano equatoriale di un dipolo}
Si consideri il campo elettrico in un punto $P$ posto sull'asse $y$. I due contributi di campo sono $\vec{E}_+$, dovuto alla carica positiva, ed $\vec{E}_-$, dovuto alla carica negativa:
\[\vec{E}_+ = \frac{1}{4 \pi \epsilon_0} \cdot \frac{q}{r_+^2} \hat{r}_+ \hspace{1em} \text{ e } \hspace{1em} \vec{E}_- = \frac{1}{4 \pi \epsilon_0} \cdot \frac{-q}{r_-^2} \hat{r}_-\]
La distanza $r_+$ tra $+q$ e $P$ è uguale alla distanza $r_-$ tra $-q$ e $P$ ed è $r_+ = r_- = r = \sqrt{y^2+a^2}$. Com'è evidente, il vettore $\vec{r}_+ = y \hat{j} - a \hat{k}$, per cui il versore $\hat{r}_+$ è la normalizzazione del vettore $\vec{r}_+$ è
\[\hat{r}_+=\frac{y \hat{j} - a \hat{k}}{r}\]
per cui si ottiene
\[\vec{E}_+=\frac{1}{4 \pi \epsilon_0} \frac{q}{r^2} = \frac{y \hat{j} - a \hat{k}}{r} = \frac{1}{4 \pi \epsilon_0 r^3} \cdot (y \hat{j} - a \hat{k})\]
... continua ...
\[\vec E = \vec{E}_+ + \vec{E}_- = \frac{1}{4 \pi \epsilon_0} \cdot \left(\frac{q}{(z-a)^2} - \frac{q}{(z+a)^2}\right)\]

\vspace{1em}
\noindent
\subsection{Campo elettrico generato da distribuzioni continue di carica}
Il campo elettrico infinitesimo $d \vec{E}$ generato da $dq$ è 
\[d \vec{E} = \frac{1}{4 \pi \epsilon_0} \frac{dq}{r^2} \hat{r}\]
per cui
\[\vec E = \frac{1}{4 \pi \epsilon_0} \int \frac{dq}{r^2} \hat{r}\]
Per cui se si considera un oggetto con distribuzione volumetrica di carica costante, si ottiene:
\[\vec E = \frac{1}{4 \pi \epsilon_0} \iiint \frac{\rho}{r^2} \hat{r} dv\]
mentre per un oggetto con distribuzione superficiale di carica costante, si ottiene:
\[\vec E = \frac{1}{4 \pi \epsilon_0} \int \int \frac{\sigma}{r^2} \hat{r} da\]
e infine, per un oggetto con distribuzione lineare di carica costante, si ottiene:
\[\vec E = \frac{1}{4 \pi \epsilon_0} \int \frac{\lambda}{r^2} \hat{r} dl\]

\vspace{1em}
\noindent
\subsection{Linee di forza del campo elettrico}
Le linee di forza d campo elettrico aiutano a farsi un'idea intuitiva del campo: sostanzialmente sono una mappa del campo. Benché le linee di forza vengano tracciate su un foglio di carta o su una lavagna (che sono bidimensionali), esse vanno immaginate nello spazio tridimensionale e sono estremamente utili, dal punto di vista grafico, anche per descrivere i campi magnetici.\\
Il concetto di linea di forza fu introdotto dal grande fisico sperimentale inglese Michael Faraday (1791-1867): ciascuna linea viene tracciata in modo che in ogni suo punto, il vettore campo elettrico $\vec E$ sia tangente alla linea stessa, cosicché le linee di forza indicano la direzione, mentre le frecce indicano il verso del campo.\\
Per esempio, in prossimità di una carica puntiforme, le linee di forza sono radiali e hanno \textbf{verso uscente da una carica positiva} ed \textbf{entrante in una carica negativa}.\\
In una data rappresentazione, la densità di linee di forza per unità di superficie dipende dal modulo del campo. Nelle regioni in cui le linee sono vicine, o fitte,$E$ è grande, mentre dove sono rade $E$ è piccolo. La densità delle linee di forza è proporzionale a $E$ e ciò può essere dimostrato tramite la legge di Gauss.\\
Dal momento che la densità delle linee per unità di superficie è proporzionale a $E$, il numero delle linee uscenti da una carica positiva o entranti in una carica negativa e proporzionale a $\vert q \vert$.\\
Un campo uniforme è rappresentato da linee di forza equidistanti, rettilinee e parallele. Il campo in prossimità di un disco in carico modo uniforme, ma lontano dal suo bordo, è pressoché uniforme.\\
Per quanto riguarda le linee di forza del campo di un disco caricato uniformemente, si evince vicino al disco e lontano dal suo bordo le linee sono tracciate in modo da apparire approssimativamente equidistanti, rettilinee e parallele.

\newpage
\noindent
\begin{center}
  5 Ottobre 2022
\end{center}
Dopo aver introdotto la definizione di campo elettrico, sono state esposte le formule di calcolo del campo elettrico a seconda della natura dell'entità che si sta studiando, come un dipolo o una carica puntiforme.

\vspace{1em}
\noindent
\subsection{Campo elettrico di una distribuzione lineare di carica}
Quando una distribuzione di carica è lunga e sottile, come accade con una carica localizzata su un filo, si parla di \textbf{distribuzione lineare di carica}.\\
La densità lineare di carica $\lambda$ è calcolata come
\[\lambda = \frac{Q}{2 l}\]
In particolare, si ha che il campo elettrico infinitesimo, in modulo vale
\[d E = \frac{1}{4 \pi \epsilon_0} \cdot \frac{dq}{y^2 + z^2}\]
Dovendo descrivere il campo elettrico come vettore, si ha
\[d \vec{E} = dE_y \hat{j} + d E_z \hat{k} = (d E \cos(\theta)) \hat{j} - (d E \sin(\theta)) \hat{k}\]
Appare evidente, dal punto di vista trigonometrico, come
\[\sin(\theta) = \frac{z}{\sqrt{y^2 + z^2}} \hspace{1em} \text{e} \hspace{1em} \cos(\theta) = \frac{y}{\sqrt{y^2 + z^2}}\]
Integrando rispetto alla componente $y$ del campo, si ottiene
\[E_y = \int d E_y = \frac{\lambda y}{4 \pi \epsilon_0} \cdot \int_{-l}^{+l} \frac{dz}{(y^2 + z^2)^{\frac{3}{2}}}\]
per cui si ottiene che
\[E_y = \frac{1}{2 \pi \epsilon_0} \cdot \frac{\lambda}{y} \cdot \frac{l}{\sqrt{l^2 + y^2}}\]
Ovviamente $E_z = 0$, in quanto per simmetria vi sono componenti del campo elettrico uguali e opposte lungo l'asse $z$.\\ 
Pertanto, generalizzando, considerando un qualunque punto del piano $xy$, osservando che $E$ deve essere simmetria azimutale rispetto all'asse $z$, analogamente a quanto si è visto per il dipolo, si otterrebbe
\[\boxed{E_R = \frac{1}{2 \pi \epsilon_0} \frac{\lambda}{R} \cdot \frac{l}{\sqrt{l^2+R^2}}}\]
dove $R=\sqrt{x^2+y^2}$.
Supponendo che $l >> R$, è chiaro che
\[\boxed{E_R = \frac{\lambda}{2 \pi \epsilon_0 R}}\]

\vspace{1em}
\subsection{Campo elettrico sull'asse di un anello carico}
Dovendo determinare $\vec E$ nei punti posti lungo l'asse di un anello circolare carico di raggio $a$ e carica $Q$. La distribuzione di carica sull'anello è uniforme e sufficientemente sottile per poter essere considerata lineare, analogamente alla distribuzione di massa di un anello.\\
Considerando un anello su un piano $yz$, il campo elettrico infinitesimo $d \vec E$ generato dalla carica $dq$ può essere decomposto nelle sue componenti $dE_x$, parallela all'asse $x$ e $dE_\perp$, perpendicolare all'asse $x$. La simmetria della distribuzione di carica richiede che $\int dE_\perp = 0$, perché elementi di carica da parti opposte dell'anello producono componenti infinitesime del campo elettrico $d E_\perp$ che si elidono reciprocamente.\\
Pertanto si ha che
\[E_x = \int d E_x = \int dE \cos(\theta) = \frac{1}{4 \pi \epsilon} \int \frac{dq}{r^2} \cos(\theta)\]
Ovviamente $\theta$ e $r^2$ rimangono costanti per ciascun elemento di carica $dq$, da cui
\[E_x = \frac{\cos(\theta)}{4 \pi \epsilon_0 r^2} \int dq = \frac{Q \cos(\theta)}{4 \pi \epsilon_0 r^2}\]
Considerando $P$ sull'asse dell'anello a distanza $x$ dal centro dell'anello stesso, si ha che
\[\cos(\theta) = \frac{x}{\sqrt{x^2 + a^2}}\]
per cui si ha
\[\boxed{E_x=\frac{Qx}{4 \pi \epsilon_0 \cdot \left(x^2+a^2\right)^{\frac{3}{2}}}}\]
 
\vspace{1em}
\noindent
\subsection{Campo elettrico di una distribuzione superficiale di carica}
Dal momento che la distribuzione di carica è uniforme e ha la forma di un disco sottile, si può trattare come una distribuzione superficiale con $\sigma = \frac{Q}{\pi R_0^2}$, dove $\pi R_0^2$ è l'area del disco.\\
È noto che $dE_x$ sull'asse di un anello di raggio $a$ e carica $dq = \sigma 2 \pi a da$ è
\[dE_x = \frac{\left(\sigma 2 \pi a da\right) x}{4 \pi \epsilon_0 \cdot \left(x^2 + a^2\right)^{\frac{3}{2}}}\]
Integrare tale espressione tra $a=0$ e $a=R_0$ equivale a sommare tutti i contributi a $E_x$ dovuti ai singoli anelli di raggio $a$ compreso tra $a=0$ e $a=R_0$:
\[E_x = \frac{2 \pi \sigma x}{4 \pi \epsilon_0} \cdot \int_0^{R_0} \frac{a da}{\left(x^2 + a^2\right)^{\frac{3}{2}}}\]
Il calcolo dell'integrale produce:
\[\boxed{E_x = \frac{\sigma x}{2 \epsilon_0} \cdot \left(\frac{1}{\sqrt{x^2}} - \frac{1}{\sqrt{x^2 + R_0^2}}\right)}\]
Dal momento che la quantità tra parentesi è sempre positiva, il segno algebrico di $E_x$ è lo stesso di $x$. Nell'ipotesi in cui $x << R_0$, si ha che
\[\boxed{E_x = \frac{\sigma}{2 \epsilon_0} \cdot \frac{x}{\vert x \vert}}\]
in cui risulta fondamentale il segno di $x$ per definire il segno del campo.

\vspace{1em}
\subsection{Particelle cariche in un campo elettrico uniforme}
Un tubo a raggi catodici è uno strumento in cui gli elettroni vengono prima accelerati e poi deflessi. In particolare, gli elettroni vengono emessi da un filamento reso incandescente e accelerati da un campo elettrico orizzontale generato da delle placche cariche nel cosiddetto \quotes{cannone elettronico}.\\
Se la forza elettrica è l'unica forza significativa che agisce sulla particelle $q \cdot \vec E$ è la forza risultante e la seconda legge di Newton fornisce
\[q \cdot \vec E = m \cdot \vec a \hspace{1em} \text{ ossia } \hspace{1em} \vec a = \frac{q \cdot \vec E}{m}\]
Decomponendo il vettore accelerazione nelle sue due componenti $x$ e $y$, è possibile ottenere
\[a_y = \frac{q E}{m} \hspace{1em} \text{ e } \hspace{1em} a_x=0 \hspace{1em} \text{ e } \hspace{1em} a_z=0\]
\[v_y = \left(\frac{qE}{m}\right) \cdot t \hspace{1em} \text{ e } \hspace{1em} v_x=v_0 \hspace{1em} \text{ e } \hspace{1em} v_z=0\]
\[y=\frac{1}{2} \left(\frac{q E}{m}\right) \cdot t^2 \hspace{1em} \text{ e } \hspace{1em} x=v_0 \cdot t \hspace{1em} \text{ e } \hspace{1em} z=0\]
È possibile ottenere anche la formula per la traiettoria parabolica della particella, ricavando $t$ come $t=\frac{x}{v_0}$, per cui
\[\boxed{y=\frac{1}{2} \frac{q E}{m v_0^2} \cdot x^2}\]

\vspace{1em}
\noindent
\subsubsection{Dipolo elettrico in campo elettrico uniforme}
Un dipolo elettrico, considerato come un sistema rigido, posto in un campo elettrico uniforme tende a ruotare in modo che il momento di dipolo risulti allineato (parallelo e concorde) al campo, soggetto alle forze esterne
\[\vec{F}_+ = q \cdot \vec E \hspace{1em} \text{ e } \hspace{1em} \vec{F}_- = -q \cdot \vec E\]
Non solo le forze interne sono nulle, ma anche le forze esterne lo sono, in quanto uguali ed opposte, mentre il momento risultante, indicato con $r_-$ il vettore posizione della carica negativa e con $r_+$ quello della carica positiva, è pari a
\[\vec \tau = \vec{r}_+ \times \vec{F}_+ + \vec{r}_- \times \vec{F}_- = \vec{r}_+ \times (+q) \cdot \vec {E} + \vec{r}_- \times (-q) \cdot \vec E = q \cdot (\vec{r}_+ - \vec{r}_-)\]
Osservando che $\vec{r}_+-\vec{r}_-$ è proprio il vettore che va dalla carica negativa a quella positiva, è possibile scrivere
\[\boxed{\vec \tau = \vec{p} \times \vec{E}}\]
Dal momento che ruotare di un angolo $d \theta$ un dipolo in un campo elettrico è necessario compiere un lavoro $dL = \tau \cdot d \theta$, è possibile associare ad ogni posizione del dipolo una certa energia potenziale, in modo tale che il lavoro compiuto all'esterno per ruotare il dipolo sia pari alla variazione di energia potenziale, da cui
\[L = \int_{\theta_1}^{\theta_2} \tau \cdot d\theta = \int_{\theta_1}^{\theta_2} pE \cdot \sin(\theta) \cdot d\theta = - p E \cos(\theta_2) - (- p E \cos(\theta_1)) = \Delta U\]
Pertanto, l'energia potenziale del dipolo in un campo elettrico può essere scritta, quindi, come
\[\boxed{U = - \vec{p} \cdot \vec{E}}\]

\noindent
\begin{center}
  10 Ottobre 2022
\end{center}
È noto che il campo elettrico è un \textbf{campo vettoriale}, definito come il rapporto tra la forza di Coulomb e una carica di prova, che deve essere piccola, ovvero
\[\vec E = \frac{\vec F}{q_0}\]
in cui la forza coulombiana si calcola come
\[\vec F = \frac{1}{4 \pi \epsilon_0} \frac{q_0 \cdot q_1}{r^2} \hat{r}\]
ove $\hat{r}$ è il versore che va da $q_0$ a $q_1$.\\
Ciò non toglie che il campo elettrico è pur sempre un vettore.

\vspace{1em}
\section{Legge di Gauss}


\vspace{1em}
\subsection{Flusso}
Alla base di \textbf{legge di Gauss} si pone la definizione di \textbf{flusso}, da considerarsi come la quantità di carica che si trova su una superficie.\\
Naturalmente, per essere molto più pratici, è possibile introdurre il concetto partendo dal campo gravitazionale, invece del campo elettrico. È intuitivo pensare che il flusso di campo gravitazionale che attraversa una superficie è dato dal \textbf{prodotto scalare} tra il valore del campo e la superficie attraversata, da cui
\[\boxed{\Phi_g = \vec g \cdot \Delta \vec S}\]
in cui $\Delta \vec S$ prende il nome di \textbf{vettore superficie}; tale vettore è così definito
\begin{itemize}
  \item il suo modulo è dato dall'area della superficie stessa;
  \item il suo orientamento è dato dall'\textbf{angolo normale} $\hat{n}$ \textbf{alla superficie stessa}, diretto verso l'alto. Tuttavia, ciò non vieta di poter definire tale angolo dalla parte opposta, ma pur sempre in \textbf{direzione esterna rispetto al volume definito dalla superficie chiusa} (infatti, una superficie chiusa è una superficie che descrive sempre un volume).
\end{itemize}
Pertanto, nel caso di un foglio rettangolare di area $A = ab$ orientato parallelamente al piano $yz$, si ha che il flusso è
\[\Phi_g = \vec g \cdot \Delta \vec S = (-g \bf{\hat j}) \cdot (+ab \bf{\hat j}) = -g \cdot ab\]

\vspace{1em}
\noindent
\textbf{Esempio 1}: Si consideri un cubo di spigolo $l$ e si calcoli il flusso di campo elettrico che attraversa ciascuna faccia, sapendo che il campo elettrico va da sinistra verso destra.\\
Ovviamente il flusso totale sarà dato dalla sommatoria del flusso che attraversa tutte e sei le facce del cubo, per cui
\[\Phi_E = \sum_{i=1}^6 \vec E \cdot \Delta \vec S\]
Tuttavia, le uniche facce che produrranno un flusso non nullo saranno quelle con vettore superficie posto parallelamente al vettore campo elettrico. Per cui
\[\Phi_E = \sum_{i=1}^6 \vec E \cdot \Delta \vec S = \vec E \cdot l^2 \cos \left(0\right) + \vec E \cdot l^2 \cos \left(\pi\right) = E \cdot l^2 - E \cdot l^2 = 0\]

\vspace{1em}
\noindent
\textbf{Esempio 2}: Si consideri un cuneo immersa in un campo uniforme $\vec E = (600 \text{N} / \text{C}) \bf{\hat i}$ e si calcoli il flusso che attraversa ciascuna delle $5$ facce e, successivamente, il flusso totale.\\
In particolare si ha che
\begin{enumerate}
  \item $\Phi_1 = E \cdot \Delta S \cdot \cos(\pi) = -600 \cdot 9 = -5400 \dfrac{\text{N} \cdot \text{m}^2}{\text{C}}$
  \item $\Phi_2 = E \cdot \Delta S \cdot \frac{3}{5} = 600 \cdot 15 \cdot \frac{3}{5} = 5400 \dfrac{\text{N} \cdot \text{m}^2}{\text{C}}$
  \item $\Phi_3,\Phi_4,\Phi_5=0\dfrac{\text{N} \cdot \text{m}^2}{\text{C}}$
\end{enumerate}
Ciò porta a concludere che il flusso totale sia nullo.

\vspace{2em}
\noindent
\textbf{Osservazione}: Gli esempi di flusso fin qui analizzati riguardano campi uniformi e superfici piane. Quando la superficie è curva, o quando il campo elettrico varia da punto a punto su di essa, il flusso si calcola dividendo quest'ultima in piccoli elementi di superficie, ciascuno abbastanza piccolo da poter essere considerato piano e tale che su di esso la variazione del campo elettrico sia trascurabile. Il flusso attraverso l'intera superficie è allora la somma dei singoli contributi dovuti a ciascuno dei piccoli elementi di superficie. Facendo tendere a zero le dimensioni di ciascun elemento e a infinito il loro numero, la somma diventa un integrale:
\[\boxed{\Phi_E = \lim_{\Delta S_i \to 0} \sum_i \vec{E}_i \cdot \Delta \vec{S}_i = \int \int_S \vec{E} \cdot \dif \vec{S}}\]
oppure, in modo analogo,
\[\Phi_E = \int \int_S  E \cos(\theta) \dif \vec{S}\]
Nel caso di una superficie chiusa, come nella maggior parte degli esempi che si tratteranno
%\[\Phi_E = \oint \oint \vec{E} \dif \vec{S}\]
in cui la notazione sul doppio integrale sta a significare che si tratta di uan superficie chiusa, ossia una superficie che definisce un volume.

\vspace{1em}
\subsection{Legge di Gauss}
Di seguito si espone la definizione di \textbf{legge di Gauss}:

% Tabella per le definizione di concetti, etc...
\vspace{1em}
\rowcolors{1}{black!5}{black!5}
\setlength{\tabcolsep}{14pt}
\renewcommand{\arraystretch}{2}
\noindent
\begin{tabularx}{\textwidth}{@{}|P|@{}}
    \hline
    {\textbf{LEGGE DI GAUSS}}\\
    \parbox{\linewidth}{Il flusso del campo elettrico attraverso una superficie chiusa arbitraria è pari alla somma algebrica delle cariche contenute all'interno del volume delimitato dalla superficie divisa per la costante dielettrica del vuoto.\\
    Sotto forma di equazione è
    \[\boxed{\Phi_E = \dfrac{Q_{\text{int}}}{\epsilon_0} \hspace{1em} \text{ovvero} \hspace{1em} \underset{S_\text{chiusa}}{\oint \oint} \vec{E} \cdot \dif \vec{S} = \frac{1}{\epsilon_0} \underset{V_\text{int}}{\iiint} \rho \dif v}\]
    \vspace{3mm}}\\
    \hline
\end{tabularx}

\vspace{2em}
\noindent
\textbf{Esempio}: Si consideri una sfera al cui centro si trova una carica $q$. Allora, applicando la legge di Gauss, si ottiene che
\[\Phi_E = \underset{\text{Sfera}}{\oint \oint} \vec{E} \cdot \dif \vec{S} = \underset{\text{Sfera}}{\oint \oint} E_r \cdot \dif S = E_r \cdot \underset{\text{Sfera}}{\oint \oint} \dif S = E_r \cdot (4 \pi r^2)\]
Dato che la carica totale presente nella sfera gaussiana è $Q_{\text{int}} = q$, la legge di Gauss porta a
\[E_r \cdot \left(4 \pi r^2\right) = \frac{q}{\epsilon_0} \hspace{1em} \text{ossia} \hspace{1em} E_r = \frac{q}{4 \pi \epsilon_0 r^2}\]
e tenendo conto della direzione del campo si ottiene la formula del campo elettrico già desunta tramite la Legge di Coulomb:
\[E_r = \frac{q}{4 \pi \epsilon_0 r^2} \cdot \bf{\hat{r}}\]

\vspace{1em}
\subsection{Deduzione della legge di Gauss dalla legge di Coulomb}
Si considerino due sfere gaussiane di raggi differenti e aventi centro comune, in corrispondenza di una stessa particella carica. Per il teorema di Gauss, i flussi attraverso le due superfici sono uguali. Dal punto di vista grafico, si può dire che il flusso è proporzionale al numero delle linee che le attraversano. Se il flusso è lo stesso, mentre la superficie è differente, ciò che varia è la densità superficiale di carica, la quale sarà maggiore per la sfera interna, di raggio inferiore, rispetto alla sfera esterna.

\newpage
\noindent
\begin{center}
  11 Ottobre 2022
\end{center}
Se il flusso di campo elettrico su una superficie chiusa è nullo, ciò non significa che il campo elettrico su ogni \quotes{faccia} dell'oggetto sia nullo.\\
Per esempio, considerato un dipolo elettrico, con una superficie posta attorno a $+q$ (denotata con $S_1$) una tra le due cariche (denotata con $S_2$), si ottiene
\[\Phi_{S_1} = \frac{+q}{\epsilon_0} \hspace{1em} \text{e} \hspace{1em} \Phi_{S_2} = 0\]
È noto che la Legge di Gauss e la Legge di Coulomb si equivalgono. Tuttavia, la Legge di Gauss è molto più ampia, in quanto non si applica, come per la Legge di Coulomb, solamente a cariche ferme, ma anche a \textbf{cariche in movimento}.

\vspace{1em}
\subsection{Deduzione della legge di Gauss dalla legge di Coulomb}
Il flusso di campo elettrico generato da una particella carica attraversi una superficie a forma di blocco arrotondato. La superficie è formata da due calotte sferiche e da quattro facce piane. Il flusso attraverso ciascuna faccia piana è nullo, mentre i flussi attraverso le due calotte sferiche sono uguali e opposti, quindi il flusso relativo all'intera superficie è nullo.\\
Ovviamente, per quanto riguarda le facce piane, il vettore campo elettrico e il vettore superficie corrispondenti sono ortogonali, per cui il flusso è nullo. Per quanto riguarda le calotte sferiche, denotate con $S_1$ a distanza $r_1$ dalla carica $q$, e con $S_2$ a distanza $r_2$ dalla carica $q$. Per tale ragione, i flussi corrispondenti sono
\[\Phi_1 = \underbrace{\int \int}_{S_1} \vec E \cdot \dif \vec S = \underbrace{\int \int}_{S_1} - E \cdot \dif S = -\frac{q}{4 \pi \epsilon_0 r_1^2} \underbrace{\int \int}_{S_1} d_S = -\frac{q}{4 \pi \epsilon_0 r_1^2} \Delta S_1\]
E, analogamente per la seconda calotta, si ottiene
\[\Phi_2 = \underbrace{\int \int}_{S_2} \vec E \cdot \dif \vec S = \underbrace{\int \int}_{S_2} E \cdot \dif S = \frac{q}{4 \pi \epsilon_0 r_2^2} \underbrace{\int \int}_{S_2} d_S = -\frac{q}{4 \pi \epsilon_0 r_2^2} \Delta S_2\]
Tuttavia è evidente come
\[\frac{\Delta S_1}{\Delta S_2} = \frac{r_1^2}{r_2^2} \hspace{1em} \rightarrow \hspace{1em} \Delta S_2 = \Delta S_1 \cdot \frac{r_2^2}{r_1^2}\]
Ciò, quindi, consente di riscrivere $\Phi_2$ come
\[\Phi_2 = \frac{q}{4 \pi \epsilon_0 r_2^2} \cdot \Delta S_2 = \frac{q}{4 \pi \epsilon_0 r_2^2} \cdot \Delta S_1 \cdot \frac{r_2^2}{r_1^2} = \frac{q}{4 \pi \epsilon_0 r_1^2} \Delta S_1 = -\Phi_1\]
Ecco, quindi, che il flusso totale è
\[\Phi = \Phi_1 + \Phi_2 = \Phi_1 + (- \Phi_1) = 0\]
In questo modo si è dimostrato che il flusso complessivo attraverso una superficie arbitraria con carica puntiforme all'esterno è nullo. Ciò che risulta fondamentale è che tale risultato non è valido solamente per la particolare superficie prescelta nella dimostrazione, ma per qualsiasi superficie chiusa. Infatti, presa una qualunque superficie, è sempre possibile scomporla in delle calotte sferiche come quella considerata, in modo tale che il flusso di ciascuna sia nulla e, quindi, anche il flusso totale sarà nullo. Ciò permette di concludere che
\[\boxed{\Phi_E = 0 \hspace{1em} (\text{per una superficie chiusa arbitraria e una carica } q \text{ esterna})}\]

\vspace{1em}
\noindent
\subsection{Flusso attraverso una superficie arbitraria con carica puntiforme all'interno}
Considerando, ora, il flusso dovuto ad una carica puntiforme $q$ posta nel centro di una superficie gaussiana sferica di raggio $r$.\\
Secondo la definizione
\[\Phi_E = \oint \oint \vec E \cdot \dif \vec S = \oint \oint \frac{q}{4 \pi \epsilon_0 r^2} \cdot \dif S = \frac{q}{4 \pi \epsilon_0 r^2} \oint \oint \dif S = \frac{1}{4 \pi \epsilon_0 r^2} \cdot 4 \pi r^2 = \frac{q}{\epsilon_0}\]
in quanto la superficie di una sfera di raggio $r$ è proprio $4 \pi r^2$. Ecco che quindi si è ottenuta la legge di Gauss:
\[\boxed{\Phi_E = \frac{q}{\epsilon_0} \hspace{1em} (\text{per una superficie chiusa arbitraria e una carica } q \text{ interna})}\]

\vspace{1em}
\noindent
\textbf{Osservazione}: Il flusso del campo elettrico prodotto da una carica puntiforme al centro di una superficie gaussiana che è sostanzialmente una sfera, ma con una porzione di calotta staccata e allontanata dal centro. Ovviamente, il flusso di campo elettrico attraverso la calotta $2$ è uguale al flusso mancante per l'assenza della calotta $1$, quindi il flusso relativo all'intera superficie è identico a quello relativo ad una sfera, ossia
\[\Phi_E = \frac{1}{\epsilon_0}\]
questo è ovvio in forza del fatto che, come visto nell'esempio precedente, il flusso di campo elettrico che attraversa le due superfici sferiche di una calotta è identico, in modulo, in quanto il rapporto tra le superfici è uguale al rapporto dei quadrati dei rispettivi raggi.

\vspace{1em}
\noindent
\subsection{Flusso attraverso una superficie arbitraria con carica puntiformi all'interno e all'esterno}
Dopo aver considerato il flusso elettrico prodotto da una sola particella carica, si supponga ora che vi sia più di una particella di cui tenere conto.\\
Data una superficie qualsiasi, in cui vi sono due cariche interne $q_1$ e $q_3$, e una carica esterna $q_2$. È noto che è valido il \textbf{principio di sovrapposizione degli effetti}, per cui
\[\vec E = \vec{E}_1 + \vec{E}_2 + \vec{E}_3\]
da cui si evince che
\[\Phi_E = \underset{S_\text{chiusa}}{\oint \oint} \vec E \cdot \dif \vec S = \underset{S_\text{chiusa}}{\oint \oint} (\vec{E}_1 + \vec{E}_2 + \vec{E}_3) \cdot \dif \vec S = \underset{S_\text{chiusa}}{\oint \oint} \vec{E}_1 \cdot \dif \vec S + \underset{S_\text{chiusa}}{\oint \oint} \vec{E}_2 \cdot \dif \vec S + \underset{S_\text{chiusa}}{\oint \oint} \vec{E}_3 \cdot \dif \vec S = \frac{q_1}{\epsilon_0} + 0 + \frac{q_3}{\epsilon_0}\]
in quanto le cariche $q_1$ e $q_3$ sono interne, mentre $q_2$ è esterna, per quanto dimostrato nei due casi precedenti.\\
Ecco che, quindi, si è ottenuta la formulazione generale della Legge di Gauss:
\[\Phi_E = \frac{Q_\text{int}}{\epsilon_0} \hspace{1em} \text{ovvero} \hspace{1em} \dfrac{\sum_{S_\text{chiusa}} q_i}{\epsilon_0} \hspace{1em} \rightarrow \hspace{1em} \underset{S_\text{chiusa}}{\oint \oint} \vec E \cdot \dif \vec S = \frac{1}{\epsilon_0} \iiint \rho \dif v\]

\vspace{1em}
\noindent
\subsection{Legge di Gauss in forma differenziale}
Di seguito si espone la definizione di \textbf{divergenza di un campo vettoriale}:

% Tabella per le definizione di concetti, etc...
\vspace{1em}
\rowcolors{1}{black!5}{black!5}
\setlength{\tabcolsep}{14pt}
\renewcommand{\arraystretch}{2}
\noindent
\begin{tabularx}{\textwidth}{@{}|P|@{}}
    \hline
    {\textbf{DIVERGENZA DI UN CAMPO VETTORIALE}}\\
    \parbox{\linewidth}{La divergenza di un campo vettoriale viene definita come la somma delle derviate parziali del campo vettoriale stesso rispetto alle $3$ direzioni dello spazio tridimensionale:
    \[\boxed{\text{div} \vec E = \lim_{v \to 0} \frac{\Delta \Phi}{\Delta v} = \frac{\dif \Phi}{\dif v}}\]
    \vspace{1mm}}\\
    \hline
\end{tabularx}

\vspace{1em}
\noindent
\textbf{Esempio}: Dato il vettore campo elettrico $\vec E=(E_x,E_y,E_z)$. Si consideri un volumetto cubico di spigoli $\dif x, \dif y$ e $\dif z$. Il flusso attraverso la faccia $ABCD$ è
\[\dif \Phi_{ABCD} = E_x \dif y \dif z\]
mentre il flusso attraverso la faccia $A'B'C'D'$ 
\[\dif \Phi_{A'B'C'D'} = -E'_x \dif y \dif z\]
Poiché le due facce del cubo sono molto vicine, è possibile scrivere
\[E'_x=E_x - \frac{\partial E_x}{\partial x} \dif x \hspace{1em} \rightarrow \hspace{1em} E_x-E'_x=\frac{\partial E_x}{\partial x} \dif x\]
cosicché, essendo $\dif v = \dif x \dif y \dif z$ si ottiene
\[\dif \Phi_{ABCD} + \dif \Phi_{A'B'C'D'} = \left(E_x-E'_x\right) \dif y \dif z = \frac{\partial E_x}{\partial x} \dif v\]
Similmente per le altre due coppie di fasce si otterrà
\[\Phi_{AA'DD'} = -E_y \dif x \dif z \hspace{1em} \text{e} \hspace{1em}\Phi_{CBB'C'} = A_y \dif x \dif z\]
\[\Phi_{AA'BB'} = -E_z \dif x \dif y \hspace{1em} \text{e} \hspace{1em}\Phi_{DD'C'C} = E_z \dif x \dif y\]
In conclusione si ottiene che
\[\dif \Phi = \left(\frac{\partial E_x}{\partial x} + \frac{\partial E_y}{\partial y} + \frac{\partial E_z}{\partial z}\right) \dif v\]

\newpage
\noindent
\begin{center}
  12 Ottobre 2022
\end{center}
Si è osservato in precedenza che, se il campo elettrico è costante e si mantiene inalterato, e passa attraverso due superfici identiche, il flusso risultante sarà sempre nullo. L'unica possibilità perché esso non si azzeri è che vi sia una differenza tra le due superfici, ovverosia un gradiente.

\vspace{1em}
\noindent
\textbf{Osservazione}: La legge di Gauss, scritta in forma differenziale, si può esprimere come
\[\boxed{\Phi = \underset{\text{Superficie}}{\oint \oint} \vec E \cdot \dif \vec S = \frac{1}{\epsilon_0} \cdot \underset{\text{Volume}}{\iiint} \rho \dif v}\]
Da ciò deriva anche la definizione di \textbf{divergenza del campo elettrico} $\vec E$ come 
\[\boxed{\text{div} \vec E = \lim_{v \to 0} \frac{\Delta \Phi}{\Delta v} = \frac{\dif \Phi}{\dif v}}\]

\vspace{1em}
\noindent
\textbf{Osservazione}: La derivata parziale di una funzione $f(x,y,z)$ che dipende da $3$ variabili, per esempio, rispetto alla variabile $x$ viene definita come
\[\frac{\partial f(x_0)}{\partial x} = \lim_{\Delta x \to 0} \frac{f(x_0+\Delta x,y,z) - f(x_0,y_0,z_)}{\Delta x}\]

\vspace{1em}
\noindent
Nel caso visto in precedenza, per quanto concerne il calcolo del flusso, si era ottenuto
\[\dif \Phi_{ABCD} + \dif \Phi_{A'B'C'D'} = \left(E_x-E'_x\right) \dif y \dif z = \frac{\partial E_x}{\partial x} \dif v\]
Considerando anche le altre componenti, si ottiene che
\[\dif \Phi = \left(\frac{\partial E_x}{\partial x} + \frac{\partial E_y}{\partial y} + \frac{\partial E_z}{\partial z}\right) \dif v\]
ma siccome, per definizione, si ha che
\[\text{div } \vec E = \frac{\dif \Phi}{\dif v}\]
è immediato evincere che
\[\text{div } \vec E = \left(\frac{\partial E_x}{\partial x} + \frac{\partial E_y}{\partial y} + \frac{\partial E_z}{\partial z}\right) = \frac{\rho}{\epsilon_0}\]

\vspace{1em}
\noindent
\textbf{Osservazione}: Talvolta, si definisce la divergenza di $\vec E$ come vettore nabla $\nabla$, ossia
\[\text{div } \vec E = \nabla \cdot \vec E\]
in cui l'operatore nabla viene definito come
\[\nabla = \left(\frac{\partial}{\partial x}; \frac{\partial}{\partial y};\frac{\partial}{\partial z}\right)\]

\newpage
\noindent
\subsubsection{Campo elettrico in prossimità di una distribuzione lineare di carica molto lunga}
Si consideri un filo molto lungo e un punto $P$ a distanza $r$ da esso. Il filo elettrico è caricato in modo uniforme, con densità di carica $\lambda$.\\
Si immagini una superficie gaussiana a forma di cilindro che avvolge il filo elettrico, di raggio $r$. Ovviamente sulle due superfici circolari il flusso di campo elettrico è nullo, in quanto il campo elettrico $\vec E$ è in direzione radiale rispetto al filo, mentre i vettori superfici sono ambedue ortogonali al vettore campo elettrico.\\
Invece, il flusso sulla superficie cilindrica è dato da
\[\Phi_E = \underset{S}{\oint \oint} \vec E \cdot \dif \vec S = \underset{S}{\oint \oint} E \cdot \dif S = E \cdot \underset{S}{\oint \oint}  \dif S = E \cdot 2\pi r \cdot h\]
La carica contenuta nel cilindro è il prodotto della densità lineare di carica $\lambda$ per l'altezza $h$ del cilindro $Q_\text{int} = \lambda h$. Quindi dalla legge di Gauss si evince che
\[\Phi_E = \frac{Q_\text{int}}{\epsilon_0} \hspace{1em} \rightarrow \hspace{1em} E 2 \pi r = \frac{\lambda h}{\epsilon_0} \hspace{1em} \rightarrow \hspace{1em} E = \frac{\lambda}{2 \pi \epsilon_0 r}\]

\vspace{1em}
\noindent
\subsubsection{Campo elettrico in prossimità di una grande lamina piana carica}
Si consideri la legge di Gauss per la determinazione del campo elettrico $E$ prodotto da una grande lamina piana con densità superficiale di carica $\sigma$ uniforme, in un punto vicino alla sua superficie.\\
Dal momento che il vettore campo elettrico è diretto perpendicolarmente alla superficie, è possibile prendere in considerazione una superficie cilindrica che attraversa la superficie della lamina. In particolare, quindi, il flusso passante per la superficie del cilindro è nulla, in quanto il vettore superficie e il vettore campo elettrico sono ortogonali. Tuttavia, per quanto riguarda le superfici circolari, il vettore campo elettrico è sempre concorde con il vettore superficie, da cui
\[\Phi_E = E \cdot \Delta S + E \cdot \Delta S = 2 E \Delta S\]
La carica contenuta nel cilindro è il prodotto della densità superficiale di carica $\sigma$ per l'area della superficie circolare, ovvero $Q_\text{int} = \sigma \Delta S$. Quindi dalla legge di Gauss si evince che
\[\Phi_E = \frac{Q_\text{int}}{\epsilon_0} \hspace{1em} \rightarrow \hspace{1em} 2 E \Delta S = \frac{\sigma \Delta S}{\epsilon_0} \hspace{1em} \rightarrow \hspace{1em} E = \frac{\sigma}{2 \epsilon_0}\]

\vspace{1em}
\noindent
\subsubsection{Superficie sferica carica}
Si determini il campo elettrico $\vec E$ in punti sia interni che esterni a una sottile sfera cava uniformemente carica di raggio $r_0$ e carica $Q$. La distribuzione di carica è simile a quella della massa di una pallina da ping-pong.\\
Si consideri un punto $P_1$ interno alla sfera, di raggio $r_1$; allora, la superficie gaussiana più conveniente da impiegare per il calcolo del campo elettrico è proprio una sfera di raggio $r_1$ e centrata in modo coincidente con la sfera carica; ciò significa che la carica interna alla sfera di raggio $r_1$ è nulla, in quanto la carica è tutta distribuita sulla superficie carica della sfera di raggio $r_0$. Pertanto, è possibile affermare che $\vec E$ è nullo fintantoché $r_1 < r_0$.\\
Nell'ipotesi in cui $r_1 \geq r_0$, il flusso complessivo sarà
\[\Phi_E = \frac{Q}{\epsilon_0} \hspace{1em} \text{ e anche } \hspace{1em} \Phi_E = E \cdot S_\text{sfera} = E \cdot 4 \pi r_0^2\]
Ciò permette di concludere che
\[E = \frac{Q}{4 \pi \epsilon_0 r_0^2}\]
che è identico al campo elettrico generato da una carica puntiforme $Q$ posta al centro della pallina.

\vspace{1em}
\noindent
\subsubsection{Sfera carica in modo uniforme}
Si determini il campo elettrico $\vec E$ in punti sia interni che esterni a una sfera cava uniformemente carica di raggio $r_0$ e carica $Q$. La distribuzione di carica è simile a quella della massa di una pallina da biliardo.\\
Si consideri, dapprincipio, un punto $P_!$ esterno alla sfera, di raggio $r_1 > r_0$, allora il flusso esterno rispetto alla sfera carica sarà
\[\Phi_{r_1} = E_{r_1} \cdot 4 \pi r_1^2 = \frac{Q}{\epsilon_0} \hspace{1em} \text{da cui} \hspace{1em} E_{r_1} = \frac{Q}{4 \pi \epsilon_0 r_1^2}\]
Si consideri, invece, un punto $P_2$ interno alla sfera, di raggio $r_2 < r_0$; allora il flusso sulla sfera di raggio $r_2$ è proprio dato da
\[\Phi_{r_2} = E_{r_2} \cdot 4 \pi r_2^2\]
Per calcolare, invece, la carica interna alla sfera di raggio $r_2$ si osserva semplicemente che la densità di carica della sfera carica è
\[\rho = \dfrac{Q}{\dfrac{4}{3} \pi r_0^3}\]
per cui, per quanto riguarda una sfera di raggio $r_2<r_0$, la carica interna sarà
\[Q_\text{int} = \rho \cdot \Delta V = \dfrac{Q}{\dfrac{4}{3} \pi r_0^3} \cdot \frac{4}{3} \pi r_2^3 = Q \cdot \frac{r_2^3}{r_0^3}\]
Per cui il flusso risultante è pari a
\[\Phi_{r_2} = \frac{Q_\text{int}}{\epsilon_0} = \dfrac{Q \cdot \dfrac{r_2^3}{r_0^3}}{\epsilon_0} = E_{r_2} \cdot 4 \pi r_2^2 \hspace{1em} \rightarrow \hspace{1em} E_{r_2} = \frac{Q \cdot r_2}{4 \pi \epsilon_0 r_0^3}\]
che è ragionevole, in quanto il campo elettrico cresce linearmente con $r$ nei punti interni alla sfera carica. Se $r_2=r_0$ si ottiene il risultato già noto
\[E_{r_0}=\frac{Q}{4 \pi \epsilon_0 r_0^2}\]

\vspace{1em}
\noindent
\subsection{Proprietà elettrostatiche di un conduttore}
Si osservi facilmente che se ci fosse un campo elettrico all'interno di un conduttore, le cariche, soggette a tale campo e, quindi, alla forza di Coulomb, inizierebbero a muoversi, creando delle correnti che porterebbero a delle movimentazioni del campo elettrico. Ciò, naturalmente, non può avvenire in condizioni elettrostatiche, secondo l'ipotesi fondamentale seguente:
\[\boxed{\vec E = 0 \hspace{1em} \text{all'interno di un conduttore in condizioni statiche}}\]
Ciò significa che se si considera una qualsiasi superficie gaussiana interna al conduttore, il flusso $\Phi_E$ è anch'esso nullo. Ma per la Legge di Gauss, se il flusso è nullo, non vi sono cariche in nessun punto all'interno del conduttore; in altre parole, la densità di carica di volume $\rho$ deve essere nulla: ciò significa che tutte le cariche devono essere localizzate sulla superficie del conduttore.

\vspace{1em}
\subsubsection{Campo nelle immediate vicinanze di un conduttore}
Considerando una superficie di un conduttore, su di essa può esseri un campo elettrico, ma esso deve essere diretto necessariamente in modo ortogonale rispetto alla superficie e non tangenzialmente; infatti, se così fosse, i portatori di carica si muoverebbero lungo la superficie e non si rispetterebbero più le condizioni di equilibrio elettrostatico.\\
Se ora si considerasse un cilindro che attraversa la superficie del conduttore considerata, essendo essa una superficie gaussiana, consente di calcolare il flusso su di essa. Ma per quanto detto, essendo il campo diretto solo ortogonalmente alla superficie ed esterno al conduttore, l'unico flusso calcolabile è
\[\Phi_{E_\perp} = \frac{\sigma \Delta S}{\epsilon_0} = E_\perp \cdot \Delta S \hspace{1em} \rightarrow \hspace{1em} E_\perp = \frac{\sigma}{\epsilon_0}\]

\newpage
\noindent
\begin{center}
  17 Ottobre 2022
\end{center}
\section{Potenziale elettrico}
Il potenziale elettrico si misura in \textbf{Volt} [\textbf{V}], anche se sarebbe più corretto parlare di \textbf{Volta}, dal nome dello scienziato italiano inventore della lampadina.

\vspace{1em}
\noindent
\subsection{Energia potenziale elettrica}
Il potenziale elettrico è l'energia potenziale elettrica per unità di carica. Gli elementi essenziali per determinare un'espressione di tale grandezza sono
\begin{itemize}
  \item il campo generato da una carica puntiforme
  \item il principio di sovrapposizione
\end{itemize}

\vspace{1em}
\noindent
\subsubsection{Energia potenziale di una particella di prova nel campo di una carica puntiforme}
Una particella di prova con carica $q_0$ viene spostata lungo un arco di circonferenza con centro in una carica puntiforme $q$ fissa. Allora si ha che
\[\int_a^b \vec{F} \cdot \dif \vec l = q_0 \cdot \int_a^b \vec E \cdot \dif \vec l = 0\]
in quanto il vettore $\dif \vec l$ è perpendicolare a $\vec E$ in ogni punti del percorso, per cui il lavoro compiuto dalla forza elettrica è nullo.

\vspace{2em}
\noindent
Nel caso di una particella di prova che viene spostata lungo un percorso radiale, il lavoro è massimo, per cui
\[\int_a^b \vec F \cdot \dif \vec l = q_0 \cdot \int_a^b \left(\frac{q}{4 \pi \epsilon_0 r^2} \bf{\hat r}\right) \cdot \left(\dif r \bf{\hat r}\right) = \frac{qq_0}{4 \pi \epsilon_0} \int_a^b \frac{\dif r}{r^2} = \left[-\frac{qq_0}{4 \pi \epsilon_0} \cdot \frac{1}{r}\right]_a^b\]
per cui si conclude che il lavoro compiuto per spostare una carica da un punto $a$ a un punto $b$ in direzione radiale al campo generato dalla carica di prova $q_0$ è
\[\boxed{W=\frac{qq_0}{4 \pi \epsilon_0} \cdot \left(\frac{1}{r_a} - \frac{1}{r_b}\right)}\]

\vspace{2em}
\noindent
Se si prende in considerazione una traiettoria di forma arbitraria alla quale è sovrapposta una linea formata da un certo numero di piccoli archi e di piccoli segmenti radiali. Naturalmente, infatti, un percorso arbitrario può essere concepito come successione di un numero infinito di archi e di segmenti radiali infinitesimi.\\
Il lavoro compiuto lungo ciascun arco è nullo, e quindi il lavoro totale compiuto lungo la traiettoria di forma arbitraria è la somma dei contributi relativi ai vari segmenti radiali, ossia sempre
\[\boxed{W=\frac{qq_0}{4 \pi \epsilon_0} \cdot \left(\frac{1}{r_a} - \frac{1}{r_b}\right)}\]

\vspace{1em}
\noindent
\textbf{Osservazione}: Dal momento che il lavoro compiuto dalla forza di Coulomb è indipendente dalla traiettoria, ma solamente dalla posizione iniziale e finale, si evince che \textbf{la forza fi Coulomb è una forza conservativa}, per cui ad essa è possibile associare un'\textbf{energia potenziale}.\\
In particolare, è noto che la variazione dell'energia potenziale da $a$ a $b$ è data dall'opposto del lavoro compiuto da $a$ a $b$, ovvero
\[\boxed{\mathcal{U}_b-\mathcal{U}_a=-\int_a^b \va{F} \cdot \dif \va{l}}\]
Nel caso di cui sopra, si ottiene che
\[\mathcal{U}_b-\mathcal{U}_a = \frac{qq_0}{4 \pi \epsilon_0} \cdot \left(\frac{1}{r_b} - \frac{1}{r_a}\right)\]
Nell'ipotesi in cui ad una distanza molto grande $r_a=\infty$ si abbia $\mathcal{U}_\infty=0$, è possibile generalizzare la formula dell'energia potenziale come
\[\boxed{\mathcal{U}(r)=-\int_{\infty}^r \va{F} \cdot \dif \va{l}}\]

\vspace{1em}
\subsubsection{Energia potenziale di una particella di prova nel campo di un numero qualsiasi di cariche puntiformi}
Si supponga che la particella di prova si trovi nel campo di due cariche puntiformi $q_1$ e $q_2$. Per il principio di sovrapposizione, la forza elettrica $\vec F$ agente sulla particella di prova è
\[\vec F = q_0 \cdot \vec E = q_0 \cdot \left(\vec{E}_1+\vec{E}_2\right)\]
Il lavoro compiuto da $\vec F$ quando la particella di prova viene portata da $a$ a $b$ è
\[\int_a^b \vec F \cdot \dif \vec l = q_0 \cdot \int_a^b (\vec{E}_1+\vec{E}_2) \cdot \dif \vec l = q_0 \cdot \left[\int_a^b \vec{E}_1 \cdot \dif \vec l + \int_a^b \vec{E}_2 \cdot \dif \vec l \right]\]
Sviluppando si ottiene che
\[q_0 \cdot \left[\int_{r_a}^{r_b} \left(\frac{q_1}{4 \pi \epsilon_0} \cdot \frac{1}{r_1^2} \hat r \right) \cdot (\dif l \hat r) + \int_{r_a}^{r_b} \left(\frac{q_2}{4 \pi \epsilon_0} \cdot \frac{1}{r_2^2} \hat{r} \right) \cdot \left(\dif l \hat{r} \right) \right] = \frac{q_0q_1}{4 \pi \epsilon_0} \cdot \left[-\frac{1}{r_1}\right]_{r_a}^{r_b} + \frac{q_0q_2}{4 \pi \epsilon_0} \cdot \left[-\frac{1}{r_2}\right]_{r_a}^{r_b}\]
ottenendo
\[\frac{q}{4 \pi \epsilon_0} \cdot \left[ q_1 \cdot \left(\frac{1}{r_{1_a}} - \frac{1}{r_{1_b}}\right) + q_2 \cdot \left(\frac{1}{r_{2_a}} - \frac{1}{r_{2_b}}\right)\right]\]
Pertanto il lavoro può essere suddiviso in due contributi, ciascuno dei quali è indipendente dal cammino percorso da $a$ a $b$. Se si pone $r_a=\infty$, si evince come
\[\boxed{\mathcal{U}=\frac{q_0}{4 \pi \epsilon_0} \cdot \left(\frac{q_1}{r_1} + \frac{q_2}{r_2}\right)}\]
in cui è fondamentale osservare come, tramite l'energia potenziale elettrica, sono stati rimossi i vettori ed è possibile operare unicamente tramite gli scalari.

\newpage
\noindent
\subsection{Potenziale elettrico}
Di seguito si espone la definizione di \textbf{potenziale elettrico}:

% Tabella per le definizione di concetti, etc...
\vspace{1em}
\rowcolors{1}{black!5}{black!5}
\setlength{\tabcolsep}{14pt}
\renewcommand{\arraystretch}{2}
\noindent
\begin{tabularx}{\textwidth}{@{}|P|@{}}
    \hline
    {\textbf{POTENZIALE ELETTRICO}}\\
    \parbox{\linewidth}{Si definisce \textbf{potenziale elettrico} $V$ come
    \[\boxed{V=\frac{\mathcal{U}}{q_0}}\]
    posto $q_0$ piccolo. L'unità di misura del potenziale elettrico è il \textbf{volt} $[\text{V}]$:
    \[[\text{V}] = \dfrac{[\text{J}]}{[\text{C}]}\]
    Si osservi, in particolare, che l'energia potenziale elettrica può anche essere misurata tramite l'\textbf{elettronvolt} $[\text{eV}]$, definito come
    \[1 \text{ eV} = (1.6 \times 10^{-19} \text{ C}) \cdot (1 \text{ V}) = 1.6 \times 10^{-19} \text{ J}\]
    \vspace{3mm}}\\
    \hline
\end{tabularx}

\vspace{2em}
\noindent
\textbf{Esercizio}: In un atomo, la distanza tipica tra un elettrone e il nucleo è di circa $1 \times 10^{-10}$ m. Si determini il potenziale prodotto da un nucleo di ossigeno in un punto che si trova alla distanza di $1.0 \times 10^{-10}$ m dal nucleo.\\
Il potenziale è facilmente calcolabile come
\[V = \frac{\mathcal{U}}{q} = \frac{1}{4 \pi \epsilon_0} \cdot \frac{q}{r} = 9.0 \times 10^9 \frac{\text{N m}^2}{\text{C}^2} \cdot \frac{8 \cdot 1.6 \times 10^{-19} \text{ C}}{1.0 \times 10^{-10} \text{ m}} = 120 \text{ V}\]
L'energia potenziale, riferita a $U_{\infty}=0$ di un elettrone in questo punto in eV si calcolerà come
\[\mathcal{U}_e = (-e) \cdot V = -120 \text{ eV}\]
che, in Joule, corrisponderà
\[\mathcal{U}_e = (-e) \cdot V = -1.6 \times 10^{-19} \cdot 120 = -1.9 \times 10^{-17} \text{ J}\]

\vspace{1em}
\subsection{Potenziale prodotto da distribuzioni continue di cariche}
Il potenziale prodotto da una distribuzione continua di carica, semplicemente dividendo la distribuzione continua di cariche infinitesime $\dif q$ e, con tale passaggio al limite, la sommatoria diviene un integrale, ossia
\[V=\frac{1}{4 \pi \epsilon_0} \cdot \lim_{
\rowcolors{1}{white}{white}
\setlength{\tabcolsep}{1pt}
\renewcommand{\arraystretch}{1}
\begin{array}{l}
  N \to +\infty\\
  q_i \to 0  
\end{array}  
} \sum_{i=1}^N \frac{q_i}{r_i} = \frac{1}{4 \pi \epsilon_0} \underset{\text{corpo carico}}{\int} \frac{\dif q}{r}\]

\vspace{2em}
\noindent
\textbf{Esercizio}: Il potenziale in un punto $P$ posto sul vertice di un quadrato di lato $93 \text{ mm}$, dove $q_1=33$ nC, $q_2=-51$ nC e $q_3=47$ nC. In base alla formula di cui sopra si ha che
\[V=\frac{1}{4 \pi \epsilon_0} \cdot \left(\frac{q_1}{r_1}+\frac{q_2}{r_2}+\frac{q_3}{r_3}\right) = 9.0 \times 10^9 \frac{\text{N m}^2}{\text{C}^2} \cdot \left(\frac{33 \times 10^{-9} \text{ C}}{93 \times 10^{-3} \text{ m}} - \frac{51 \times 10^{-9} \text{ C}}{\sqrt{2} \cdot 93 \times 10^{-3} \text{ m}} + \frac{47 \times 10^{-9} \text{ C}}{93 \times 10^{-3} \text{ m}}\right)\]
Il valore che si ottiene eseguendo i calcoli è $V= 4.2 \times 10^3$ V.

\newpage
\noindent
\begin{center}
  18 Ottobre 2022
\end{center}
\subsection{Differenza di potenziale}
Se $\mathcal{U}_b - \mathcal{U}_a$ rappresenta la variazione di energia potenziale elettrica di una particella di prova con carica $q_0$ quando viene spostata dal punto $a$ al punto $b$, la differenza di potenziale tra i punti $a$ e $b$ è definita come
\[\boxed{\bf{V_b-V_a=\frac{\mathcal{U}_b-\mathcal{U}_a}{q_0}}}\]
Inoltre, dal momento che la variazione di energia potenziale elettrica di una particella di prova con carica $q_0$ è l'opposto del lavoro compiuto dalla forza elettrica, si ha
\[\mathcal{U}_b-\mathcal{U}_a=-q_0 \cdot \int_a^b \vec E \cdot \dif \vec l\]
dividendo per $q_0$ si ottiene la differenza di potenziale in funzione del campo elettrico:
\[\boxed{\bf{V_b-V_a=-\int_a^b \vec E \cdot \dif \vec l}}\]

\vspace{1em}
\noindent
\textbf{Esempio}: Si determini la differenza di potenziale $V_b-V_a$ in un campo uniforme orientato nella direzione $+x$. Si procede in questo modo
\[V_b-V_a=-\int_a^b \vec E \cdot \dif \vec l = -\int_a^b (E {\bf{\hat i}}) \cdot (\dif x {\bf{\hat i}}) = -\int_a^b E \cdot \dif l = -E \cdot (x_b-x_a) = -E \cdot \Delta x\]
Nell'ipotesi, quindi, che $V_0$ rappresenti il potenziale dei punti del piano $yz$, posto $x=0$ e che $V(x)$ rappresenti il potenziale dei punti del piano parallelo al piano $yz$ con coordinata $x$. Ciò fornisce
\[\boxed{V(x) = V_0-Ex}\]

\vspace{1em}
\noindent
\textbf{Esempio}: Si considerino due placche cariche (anodo-catodo) in cui la differenza di energia potenziale è pari a
\[\Delta V = V_A-V_C=2500 \text{ V}\]
Naturalmente, per la conservazione dell'energia si ottiene che
\[\mathcal{U}_A+\mathcal{K}_A=\mathcal{U}_A+\mathcal{K}_A\]
Preso un elettrone che parte dal catodo e raggiunge l'anodo, è ovvio che $\mathcal{K}_C=0$, per cui
\[K_A=U_C-U_A \hspace{1em} \rightarrow \hspace{1em} \frac{1}{2}mv^2 = - q_0 \cdot \Delta V\]
per cui la velocità cercata è
\[v = \sqrt{\frac{2 \cdot (-e) \cdot (-\Delta V)}{m}} = \sqrt{\frac{2 \cdot 1.6 \times 10^{-19} \text{ C} \cdot 2500 \text{V}}{9.1 \times 10^{-31} \text{ kg}}} = 3.0 \times 10^7 \frac{\text{m}}{\text{s}}\]

\vspace{1em}
\noindent
\textbf{Esercizio}: Si determini il potenziale nei punti dell'asse di un anello circolare uniformemente carico di raggio $a$ e di carica totale $Q$. L'anello è così sottile da poter essere considerato una distribuzione lineare di carica. Pertanto si ottiene che
\[V=\frac{1}{4 \pi \epsilon_0} \cdot \int \frac{\dif q}{r}\]
Per la configurazione del punto considerato, si ha che
\[r=\sqrt{a^2+x^2}\]
per cui si evince che
\[V=\frac{1}{4 \pi \epsilon_0} \cdot \int \frac{\dif q}{\sqrt{x^2+a^2}}\]
L'integrazione comporta la somma dei contributi di potenziale dovuti a tutti gli elementi di carica $\dif q$ che formano l'anello. Siccome $x$ e $a$ sono costanti rispetto a tale integrazione, si ottiene
\[V= \frac{1}{4 \pi \epsilon_0} \cdot \frac{Q}{\sqrt{x^2+a^2}}\]
Tale esempio permette di capire come il potenziale diminuisca con l'aumento della distanza $x$, sia essa positiva o negativa.

\vspace{1em}
\noindent
\subsection{Relazione tra campo e potenziale elettrico}
Il potenziale, con la posizione di riferimento in un punto a grande distanza, può essere ottenuto dall'equazione
\[V=-\int_a^b\vec E \cdot \dif \vec l\]
facendo corrispondere il punto $a$ a $r=\infty$, tale per cui $V_a=V_r=\infty$ e il punto $b$ alla posizione $P$ in cui si valuta il potenziale, con $V_b=V$. Ciò fornisce
\[V=-\int_\infty^P \vec E \cdot \dif \vec l\]
Si supponga, allora, di impiegare l'equazione di partenza per determinare la differenza di potenziale tra due punti contigui $a=(x,y,z)$ e $b=(x+\Delta x,y,z)$. Dal momento che i due punti hanno le stesse coordinate $y$ e $z$, si integrerà lungo una linea parallela all'asse $x$, per cui $\dif \vec l = \dif x' {\bf{\hat i}}$. Quindi
\[\vec E \cdot \dif \vec l = (E_x {\bf{\hat i}} + E_y {\bf{\hat j}} + E_z {\bf{\hat k}}) \cdot (\dif x' {\bf{\hat i}}) = E_x \cdot \dif x'\]
per cui si evince che
\[V(x+\Delta x,y,z)-V(x,y,z)=-\int_x^{x+\Delta x} E_x \cdot \dif x'\]
Supponendo che $E_x$ sia pressoché costante tra $x$ e $x+\Delta x$. Ciò significa che $E_x$ viene portato fuori dal segno di integrale, ottenendo
\[-\int_x^{x+\Delta x} E_x \cdot \dif x' = - E_x \cdot \int_x^{x+\Delta x} \dif x' = -E_x \cdot \Delta x\]
Ciò permette di concludere che
\[\lim_{\Delta x \to 0} \left[\frac{V(x+\Delta x,y,z)-V(x,y,z)}{\Delta x}\right]=-E_x\]
Tale grandezza viene rappresentata con il simbolo
\[\frac{\partial V}{\partial x}\]
ed è definita \textbf{derivata parziale di $V$ rispetto a $x$}, per cui, osservando che variazioni infinitesime del potenziale nelle direzioni $y$ e $z$ forniscono risultati analoghi, per cui
\[E_x=-\frac{\partial V}{\partial x} \hspace{1em} E_y=-\frac{\partial V}{\partial y} \hspace{1em} E_z=-\frac{\partial V}{\partial z}\]
o, in maniera più sintetica:
\[\boxed{\vec E = - \left(\frac{\partial V}{\partial x} {\bf{\hat i}} + \frac{\partial V}{\partial y} {\bf{\hat j}} + \frac{\partial V}{\partial z} {\bf{\hat k}}\right)}\]

\vspace{1em}
\noindent
\textbf{Osservazione}: Se si conosce un'espressione del potenziale $V$ dovuto ad una distribuzione di carica, si può impiegare l'espressione di cui sopra per determinare $\vec E$. Per esempio, nel caso del potenziale $V$ prodotto da una carica puntiforme, per trovare il campo $\vec E$ generato da una carica puntiforme come segue:
\[E_r = - \frac{d}{dr} \frac{q}{4 pi \epsilon_0 r} = -\frac{q}{4 \pi \epsilon_0} \cdot \left(-\frac{1}{r^2}\right) = \frac{q}{4 \pi \epsilon_0 r^2}\]

\vspace{1em}
\noindent
\textbf{Esempio 1}: Se, a titolo di esempio, si considera una generica espressione per il potenziale elettrico, come esposto di seguito:
\[V=ax^2+y+3z\]
Allora il campo elettrico è
\[\vec E = - \left(\frac{\partial V}{\partial x} {\bf{\hat i}} + \frac{\partial V}{\partial y} {\bf{\hat j}} + \frac{\partial V}{\partial z} {\bf{\hat k}}\right) = - \left(2ax {\bf{\hat i}} + 1 {\bf{\hat j}} + 3 {\bf{\hat k}}\right)\]

\vspace{1em}
\noindent
\textbf{Esempio 2}: Se, a titolo di esempio, si considera una generica espressione per il potenziale elettrico, come esposto di seguito:
\[V=l \cdot \frac{Q}{\sqrt{x^2+a^2}}\]
Allora il campo elettrico è
\[\vec E = - \left(\frac{\partial V}{\partial x} {\bf{\hat i}} + \frac{\partial V}{\partial y} {\bf{\hat j}} + \frac{\partial V}{\partial z} {\bf{\hat k}}\right) = - \left(-k\frac{Q}{\sqrt{(x^2+a^2)^3}} x {\bf{\hat i}} + 0 {\bf{\hat j}} + 0 {\bf{\hat k}}\right)\]

\vspace{1em}
\noindent
\subsubsection{Campo elettrico come gradiente del potenziale}
Indicato con nabla $\nabla$ il seguente operatore vettoriale
\[\nabla = \left(\frac{\partial}{\partial x} {\bf{\hat i}} + \frac{\partial}{\partial y} {\bf{\hat j}} + \frac{\partial}{\partial z} {\bf{\hat k}}\right)\]
si può scrivere quanto esposto in precedenza come segue
\[\boxed{\vec E = - \nabla V} \hspace{1em} \text{e} \hspace{1em} \boxed{\vec E = -\text{grad } V}\]

\vspace{1em}
\noindent
\subsubsection{Equazioni di Laplace e di Poisson}
Siccome per la \textbf{legge di Gauss} è vero che
\[\text{div } \vec E = \left(\frac{\partial E_x}{\partial x} + \frac{\partial E_y}{\partial y} + \frac{\partial E_z}{\partial z}\right) = \frac{\rho}{\epsilon_0}\]
per quanto osservato in precedenza si ha che
\[E_x=-\frac{\partial V}{\partial x} \hspace{1em} E_y=-\frac{\partial V}{\partial y} \hspace{1em} E_z=-\frac{\partial V}{\partial z}\]
per cui
\[\frac{\partial}{\partial x} \left(-\frac{\partial V}{\partial x}\right) + \frac{\partial}{\partial y} \left(-\frac{\partial V}{\partial y}\right) + \frac{\partial}{\partial z} \left(-\frac{\partial V}{\partial z}\right) = \frac{\rho}{\epsilon_0}\]
che permette di affermare che la funzione potenziale deve verificare
\begin{itemize}
  \item l'\textbf{equazione di Poisson}
  \[\frac{\partial^2 V}{\partial x^2} + \frac{\partial^2 V}{\partial y^2} + \frac{\partial^2 V}{\partial z^2} = \frac{\rho}{\epsilon_0}\]
  che può essere riscritto come
  \[\text{div grad } V = -\frac{\rho}{\epsilon_0} \hspace{1em} \text{oppure} \hspace{1em} \nabla^2 V = -\frac{\rho}{\epsilon_0}\]

  \item e, in particolare, nelle regioni di spazio prive di cariche elettriche, l'\textbf{equazione di Laplace}
  \[\frac{\partial^2 V}{\partial x^2} + \frac{\partial^2 V}{\partial y^2} + \frac{\partial^2 V}{\partial z^2} = 0\]
  che può essere riscritto come
  \[\text{div grad } V = 0 \hspace{1em} \text{oppure} \hspace{1em} \nabla^2 V = 0\]
\end{itemize}

\vspace{1em}
\noindent
\textbf{Osservazione}: Si determini l'energia potenziale di un dipolo in un campo elettrico uniforme diretto parallelamente all'asse $x$.\\
Per come è stato rappresentato graficamente il dipolo, è immediato evincere che il posizionamento della carica positiva e negativa sia
\begin{itemize}
  \item $x_+ = a + a \cos(\theta)$
  \item $x_- = a - a \cos(\theta)$
\end{itemize}
Pertanto l'energia potenziale di ambedue le cariche si calcolerà come segue
\begin{itemize}
  \item $\mathcal{U}_+ = q \cdot V_+ = q \cdot \left(-E \cdot (x_0 + a \cos(\theta)) + V_0\right)$
  \item $\mathcal{U}_- = -q \cdot V_-  = -q \cdot \left(-E \cdot (x_0 - a \cos(\theta)) + V_0\right)$
\end{itemize}
Pertanto l'energia potenziale complessiva $\mathcal{U}$ del dipolo è
\[\mathcal{U}=\mathcal{U}_++\mathcal{U}_-=q \cdot V_+ = x_+ = q \cdot \left(-E \cdot (x_0 + a \cos(\theta)) + V_0\right)-q \cdot \left(-E \cdot (x_0 - a \cos(\theta)) + V_0\right)\]
ottenendo
\[\mathcal{U}=\mathcal{U}_++\mathcal{U}_-=-2aqE\cos(\theta)=-pE\cos(\theta)\]
che permette di concludere che
\[\boxed{\mathcal{U} = - \va{p} \cdot \va{E}}\]

\newpage
\begin{center}
  19 Ottobre 2022
\end{center}
È noto che il campo elettrico è l'opposto del gradiente del potenziale, ovverosia
\[\vec E = - \nabla V = - \left(\frac{\partial V}{\partial x} + \frac{\partial V}{\partial y} + \frac{\partial V}{\partial z}\right)\]

\vspace{2em}
\noindent
\textbf{Esercizio 1}: Si consideri il caso di due cilindri concentrici di raggio $R_1 \leq R_2$, il primo con distribuzione superficiale di carica $\sigma_1$, mentre il secondo con distribuzione superficiale di carica $\sigma_2$.\\
Il flusso di campo elettrico all'interno del cilindro più interno (\textbf{che è vuoto}), con $0 < r < R_1$ è $\Phi_1 = 0$, in quanto non c'è carica all'interno del cilindro stesso.\\
Considerando, ora, come superficie gaussiana il cilindro di raggio $R_1 \leq r < R_2$, è immediato evincere che la quantità di carica considerata è $Q_1=\sigma_1 \cdot 2 \pi r \cdot h$. Il flusso associato alla superficie gaussiana considerata è
\[\Phi_2=2\pi h r \cdot E(r) \hspace{1em} \rightarrow \hspace{1em} \vec E(r) = \frac{\Phi_2}{2 \pi h r} = \frac{\sigma_1 \cdot R_1}{\epsilon_0} \cdot \frac{1}{r} {\bf{\hat r}}\]
Per $r \geq R_2$ si ottiene che
\[E(r) \cdot 2 \pi r h = \frac{Q_1+Q_2}{\epsilon_0} = \frac{2\pi h}{\epsilon_0} \cdot \left(\sigma_1 \cdot R_1 + \sigma_2 \cdot R_2\right)\]
che può essere semplificato come
\[E(r) = \frac{2 \pi h \cdot (\sigma_1 R_1 + \sigma_2 R_2)}{\epsilon_0 2 \pi h r} \hspace{1em} \rightarrow \hspace{1em} \vec E(r) = \frac{\sigma_1 R_1 + \sigma_2 R_2}{\epsilon_0} \cdot \frac{1}{r} {\bf{\hat r}}\]
In questo modo è possibile ottenere il campo elettrico complessivo, grazie al principio di sovrapposizione.\\
Se ora si volesse calcolare il potenziale elettrico, per $r \geq R_2$, si dovrebbe calcolare il seguente integrale
\[V(r) = -\int_{0}^{R_2} \frac{\sigma_1 R_1 + \sigma_2 R_2}{\epsilon_0} \cdot \frac{1}{r} \dif r = - \frac{\sigma_1 R_1 + \sigma_2 R_2}{\epsilon_0} \cdot \int_{0}^{R_2} \frac{1}{r} \dif r = - \frac{\sigma_1 R_1 + \sigma_2 R_2}{\epsilon_0} \cdot \left[\log(r)\right]_{R_2}^{r}\]
ottenendo come risultato
\[V(r) = - \frac{\sigma_1 R_1 + \sigma_2 R_2}{\epsilon_0} \cdot \log \left(\frac{r}{R_2}\right)\]
che è coerente con quanto osservato per il campo elettrico: il potenziale è sempre negativo in quanto il campo elettrico è decrescente.\\
Per quanto riguarda la regione da $R_1 < r < R_2$ si ottiene che
\[V(R_2) - V(R_1) = -\int_{R_1}^{R_2} \frac{\sigma_1 \cdot R_1}{\epsilon_0} \cdot \frac{1}{r} \dif r = - \frac{\sigma_1 \cdot R_1}{\epsilon_0} \cdot \left[\log(r)\right] = - \frac{\sigma_1 R_1}{\epsilon_0} \cdot \log \left(\frac{R_2}{R_1}\right)\]

\vspace{1em}
\noindent
\textbf{Osservazione}: Avendo implicitamente sottinteso che $V(R_2)=0$, gli estremi di integrazione devono essere corretti rispetto a quelli impiegati. In quanto ciò che varia è la funzione di campo elettrico, per cui si ha 
\[V(r) = -\int_{r}^{R_2} \frac{\sigma_1 \cdot R_1}{\epsilon_0} \cdot \frac{1}{r} \dif r = -\frac{\sigma_1 \cdot R_1}{\epsilon_0} \cdot \log \left(\frac{r}{R_2}\right)\]
in cui è $R_1 < r < R_2$, per cui il potenziale è positivo e crescente.

\vspace{1em}
\noindent
Considerando, ora, il caso in cui $r \leq R_1$, si calcoli il potenziale seguente
\[V(r) = -\int_{R_2}^r 0 \cdot \dif r = c\]
Tale calcolo, in questo caso, visto che il campo elettrico per $r \to +\infty$ non diverge, essendo nullo, quindi può essere anche scritto come
\[- \left[V(R_1)-V(R_2)\right]\]
ma avendo assunto che $V(R_2)=0$, si ottiene esclusivamente $V(r)=-V(R_1)$ che corrisponde esattamente all'opposto della funzione potenziale 
\[V(r)-\frac{\sigma_1 \cdot R_1}{\epsilon_0} \cdot \log \left(\frac{r}{R_2}\right)\]
calcolata in $R_1$, ottenendo
\[V(r)=-V(R_1)=\frac{\sigma_1 \cdot R_1}{\epsilon_0} \cdot \log \left(\frac{R_1}{R_2}\right)\]

\vspace{2em}
\noindent
\textbf{Esercizio 2}: Si consideri una sfera solida con densità di carica di volume $\rho$ e di raggio $R_1$, con centro in $O_1$. A destra di $O_1$, in corrispondenza di $O_2$, vi è un ulteriore cavità sferica di raggio $R_2<R_1$. Si determini il campo elettrico all'interno dell foro.\\
Il campo elettrico generato dalla sfera di raggio $R_1$, al di fuori della sfera è
\[\vec E(r) = \frac{Q}{4 \pi \epsilon_0 r^2} {\bf{\hat r}}\]
Il campo elettrico della sfera più grande, trascurando il campo elettrico relativo alla cavità sferica, per $r<R_1$ è dato da
\[\frac{\Phi_\text{sfera}}{S_\text{sfera}} = \frac{Q_{\text{int}}}{\epsilon_0 \cdot S_\text{sfera}} = \frac{4}{3} \pi r^3 \cdot \rho \cdot \frac{1}{\epsilon_0 \cdot 4 \pi r^2} = \frac{\rho \cdot r}{3 \cdot \epsilon_0} {\bf{\hat r}}\]
Per conoscere il campo elettrico della cavità sferica, si può considerare come se vi fosse un campo elettrico negativo, dovuto ad una densità di carica di volume $-\rho$, per cui si ha
\[\vec{E}_+(r) = \frac{\rho \cdot r}{3 \cdot \epsilon_0} {\bf{\hat r}} \hspace{1em} \text{con } r < R_1 \hspace{3em}  \text{e} \hspace{3em} \vec{E}_- = -\frac{\rho \cdot r}{3 \cdot \epsilon_0} {\bf{\hat r}} \hspace{1em} \text{con } r < R_2\]
\end{document}